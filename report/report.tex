\documentclass{article}

\usepackage[utf8]{inputenc}
\usepackage{graphicx}
\usepackage{amsmath}
\usepackage{amsfonts}
\usepackage{amssymb}
\usepackage{comment}
\usepackage{gensymb}
\usepackage{geometry}
\usepackage{float}
\usepackage{caption}
\usepackage{subcaption}
\usepackage{listings}
\usepackage{enumitem}
\usepackage{fontspec}
\usepackage{xcolor}

\setmonofont{JetBrainsMonoNerdFont-Regular.ttf}[Path=./]

% Catppuccin Macchiato color scheme
\definecolor{ctp-macchiato-rosewater}{HTML}{f4dbd6}
\definecolor{ctp-macchiato-flamingo}{HTML}{f0c6c6}
\definecolor{ctp-macchiato-pink}{HTML}{f5bde6}
\definecolor{ctp-macchiato-mauve}{HTML}{c6a0f6}
\definecolor{ctp-macchiato-red}{HTML}{ed8796}
\definecolor{ctp-macchiato-maroon}{HTML}{ee99a0}
\definecolor{ctp-macchiato-peach}{HTML}{f5a97f}
\definecolor{ctp-macchiato-yellow}{HTML}{eed49f}
\definecolor{ctp-macchiato-green}{HTML}{a6da95}
\definecolor{ctp-macchiato-teal}{HTML}{8bd5ca}
\definecolor{ctp-macchiato-sky}{HTML}{91d7e3}
\definecolor{ctp-macchiato-sapphire}{HTML}{7dc4e4}
\definecolor{ctp-macchiato-blue}{HTML}{8aadf4}
\definecolor{ctp-macchiato-lavender}{HTML}{b7bdf8}
\definecolor{ctp-macchiato-text}{HTML}{cad3f5}
\definecolor{ctp-macchiato-subtext1}{HTML}{b8c0e0}
\definecolor{ctp-macchiato-subtext0}{HTML}{a5adcb}
\definecolor{ctp-macchiato-overlay2}{HTML}{939ab7}
\definecolor{ctp-macchiato-overlay1}{HTML}{8087a2}
\definecolor{ctp-macchiato-overlay0}{HTML}{6e738d}
\definecolor{ctp-macchiato-surface2}{HTML}{5b6078}
\definecolor{ctp-macchiato-surface1}{HTML}{494d64}
\definecolor{ctp-macchiato-surface0}{HTML}{363a4f}
\definecolor{ctp-macchiato-base}{HTML}{24273a}
\definecolor{ctp-macchiato-mantle}{HTML}{1e2030}
\definecolor{ctp-macchiato-crust}{HTML}{181926}

\lstdefinelanguage{myassembly}{
    basicstyle=\ttfamily\color{ctp-macchiato-text},
    backgroundcolor=\color{ctp-macchiato-base},
    commentstyle=\color{ctp-macchiato-surface1},
    keywordstyle=\color{ctp-macchiato-mauve},
    stringstyle=\color{ctp-macchiato-green},
    identifierstyle=\color{ctp-macchiato-blue},
    frame=single,
    rulecolor=\color{ctp-macchiato-base},
    breaklines=true,
    breakatwhitespace=true,
    breakautoindent=true,
    showspaces=false,
    showstringspaces=false,
    showtabs=false,
    tabsize=4,
    captionpos=b,
    belowskip=1em,
    belowcaptionskip=2em,
    % escapeinside={%}{)},
    morecomment=[l]{;},
    morestring=[b]",
    morekeywords={irmovq, rrmovq, mrmovq, rmmovq, pushq, popq, nop, halt, mov, add, sub, mul, div, cmp, jmp, je, jne, jg, jge, jl, jle, push, pop, call, ret, lea, xor, or, and, not, test},
    sensitive=false,
    upquote=true,
}

\lstset{
    basicstyle=\fontsize{8}{12}\ttfamily
}

\geometry{a4paper, margin=1in}

\title{CSCE 312 Lab 6 - Term Project}
\author{Rita Hernandez Guerrero, Kevin Lei}

\date{May 2, 2024}

\begin{document}

\maketitle

\section{Y86 Instruction Set Architecture}
The circuit for the Y86 Instruction Set Architecture (ISA) was implemented through Logisim. 
The circuit has five main sub-circuits: the Fetch stage, the Decode and Write Back stage, the Execute stage, the Memory stage, and the Program Counter (PC) Update stage. 
In the main circuit, on every clock cycle we check if \lstinline{icode} and \lstinline{ifun} are 0.
If they are 0, the program will halt and finish executing. 

\begin{figure}[H]
    \centering
    \includegraphics[width=0.8\textwidth]{./images/main_circuit.png}
    \caption{Main circuit for the Y86 ISA}
\end{figure}

\begin{figure}[H]
    \centering
    \includegraphics[width=0.7\textwidth]{./images/instructions.png}
    \caption{Instruction format}
\end{figure}
\newpage
\begin{figure}[H]
    \centering
    \includegraphics[width=0.7\textwidth]{./images/sub_instructions.png}
    \caption{Function specific instructions}
\end{figure}

\subsection{Fetch Stage}
In the Fetch stage, the instruction memory is implemented. 
The Fetch stage can read up to 10 bytes from memory for one instruction and the address of the first byte is given by the PC (program counter). 
Subsequently, this stage generates meaningful data and control signals from the instruction for subsequent processing stages. 
Byte 0 from the instruction memory is always the \lstinline{icode} and \lstinline{ifun} values.
Depending on \lstinline{icode}, the \lstinline{rA} and \lstinline{rB} values are selected from either byte 1 or byte 2.
Likewise, the \lstinline{valC} value is selected from bytes 2 to 9 or 1 to 8, depending on the instruction.
Finally, \lstinline{valP} is set to the next instruction address, which is the current PC value plus the length of the instruction.

\begin{figure}[H]
    \centering
    \includegraphics[width=\textwidth]{./images/fetch.png}
    \caption{Fetch stage}
\end{figure}

Within the instruction memory, a 4k $\times$ 8 RAM is used to read the instruction memory. 
The total size is 4KB because of the 12-bit address width and the 8-bit of data bit width that are both read. 
The value from the RAM is then broken up into 10 different register files, each register file representing a byte of the instruction memory line. 
A counter is set up to make sure that each line of the instruction is read.

\begin{figure}[H]
    \centering
    \includegraphics[width=0.8\textwidth]{./images/instruction_memory.png}
    \caption{Instruction memory}
\end{figure}

\subsection{Decode and Write Back Stage}

Data is read from register files in the decode stage. 
The register file has two ports for reads. 
One being \lstinline{srcA} and the other being represented as \lstinline{srcB}. 
The outputs, write port addresses, are \lstinline{dstE} and \lstinline{dstM}. 
Register IDs are selected by \lstinline{icode}.
If the fetched instruction does not have to read/write register, a default value of \lstinline{0xF} is selected. 
The module also takes into account the condition value (\lstinline{cnd}) which indicates whether or not a conditional move is needed. 
The cnd value is computed from the Execute stage. 

\begin{figure}[H]
    \centering
    \includegraphics[width=0.8\textwidth]{./images/decode_write_back.png}
    \caption{Decode and Write Back stage}
\end{figure}

Within the Decode and Write Back stage, a register file component is implemented to update the registers when needed. 
A 4x16 write decoder is used to load either \lstinline{valA} or \lstinline{valE} into a specific register, depending on \lstinline{dstM} and \lstinline{dstE}. 
Simairly, a 4x16 read decoder is used to decide whether to load either \lstinline{valA} or \lstinline{valB} into a specific register, depending on \lstinline{srcA} and \lstinline{srcB}.  

\begin{figure}[H]
    \centering
    \includegraphics[width=0.8\textwidth]{./images/register_file.png}
    \caption{Register file}
\end{figure}

\subsection{Execute Stage}

The Execution stage mainly surrounds the Arithmetic Logic Unit (ALU) performing the four operations (addq, subq, andq, and xorq). 
A condition code is also calculated in a form of (Zero Flag/Sign Flag/Overflow Flag). 
The Condition Code (CC) module lets the CC store the last condition code. 
The SetCC module generates a signal to store CC. It is set if the icode is 6 (\lstinline{OPq}). 
The signal of Cond is used by \lstinline{jXX} in NewPC, \lstinline{cmovXX} in Write Back stage. 
ALU\_fun determines the operation that must be operated by the ALU. 
ALU\_A and ALU\_B compute the inputs for the ALU. 

\begin{figure}[H]
    \centering
    \includegraphics[width=0.8\textwidth]{./images/execute.png}
    \caption{Execute stage}
\end{figure}

\begin{figure}[H]
    \centering
    \includegraphics[width=0.8\textwidth]{./images/set_cc.png}
    \caption{Set Condition Code module}
\end{figure}

As mentioned above, ALU\_A and ALU\_B both determine the inputs of the ALU depending on the instruction code given.
For example, the instruction code (\lstinline{icode}) determines if ALU\_A's output will be either \lstinline{valA}, \lstinline{valC}, 8 or -8 through a 16x1 multiplexer. 

\begin{figure}[H]
    \centering
    \includegraphics[width=0.8\textwidth]{./images/alu_a.png}
    \caption{ALU\_A module}
\end{figure}

Likewise, the output for ALU\_B was derived from a 16x1 multiplexer that chose from either 0 or \lstinline{valB}, depending on the instruction code. 

\begin{figure}[H]
    \centering
    \includegraphics[width=0.8\textwidth]{./images/alu_b.png}
    \caption{ALU\_B module}
\end{figure}

To find out which function the ALU has to perform, a 16x1 multiplexer uses the instruction code to choose between 0 (addition) or other functions determined by \lstinline{ifun} (the instruction function).

\begin{figure}[H]
    \centering
    \includegraphics[width=0.8\textwidth]{./images/alu_fun.png}
    \caption{ALU\_fun module}
\end{figure}

The condition code (CC) module stores the last condition calculated through a register that has its write enable on depending if the setCC output.

\begin{figure}[H]
    \centering
    \includegraphics[width=0.8\textwidth]{./images/cc.png}
    \caption{Condition Code module}
\end{figure}

The condition module sets the condition flag off (cnd) depending on the condition code found by the ALU and the instruction function being read. 
For example, if the instruction function were jl (ifun = 2), the expected condition code would be an input of 011 (ZF/SF/OF). 
AND gates are set in place to make sure both inputs are met. 
If both inputs meet the requirement, then cnd will output 1. 

\begin{figure}[H]
    \centering
    \includegraphics[width=0.8\textwidth]{./images/cond.png}
    \caption{Condition module}
\end{figure}

In the actual ALU module, both inputs go through a certain function determined by a 4x1 multiplexer that selects based on the ALU\_func module's output. 
The four possible functions are addition, subtraction, the bitwise AND operation, or the bitwise XOR operation. 
The condition code is determined by comparing the new value with 0. 
Overflow is also checked through the built-in adder operation. 

\begin{figure}[H]
    \centering
    \includegraphics[width=0.8\textwidth]{./images/alu.png}
    \caption{ALU module}
\end{figure}

\subsection{Memory Stage}

The main function for the Memory stage is to read or write memory word. 
The control logic surrounding this stage is the stat, mem.read, mem.write, mem.addr, and the mem.data modules. 
The stat module concludes what the status of the instruction is. 
Mem.read determines if the word should be read. 
Likewise, Mem.write determines if the word should be written. 
Mem.addr selects the address and Mem.data selects the data. 

\begin{figure}[H]
    \centering
    \includegraphics[width=0.7\textwidth]{./images/memory.png}
    \caption{Memory stage}
\end{figure}

\subsection{PC Update Stage}

The PC Update stage sets a new value for the program counter (PC). 
This new value is computed through a 16x1 multiplexer that selects based on the instruction code. 
The multiplexer determines the current PC value and the result of certain operations or conditions within the processor. 
A special case is for instruction code 7 (any jump function) which uses a 2x1 multiplexer to determine the new PC between valC or valP, depending if the condition flag is on. 

\begin{figure}[H]
    \centering
    \includegraphics[width=0.8\textwidth]{./images/pc_update.png}
    \caption{PC Update stage}
\end{figure}

\section{Timing Analysis}

The following is a timing diagram for the Y86 ISA implementation.

\begin{figure}[H]
    \centering
    \includegraphics[width=\textwidth]{./images/wavedrom.png}
    \caption{Timing diagram for the Y86 ISA implementation}
\end{figure}

\lstset{
    basicstyle=\fontsize{8}{12}\ttfamily\color{ctp-macchiato-text},
    backgroundcolor=\color{ctp-macchiato-base},
    commentstyle=\color{ctp-macchiato-surface1},
    keywordstyle=\color{ctp-macchiato-mauve},
    stringstyle=\color{ctp-macchiato-green},
    identifierstyle=\color{ctp-macchiato-blue},
    frame=single,
    rulecolor=\color{ctp-macchiato-base},
    breaklines=true,
    breakatwhitespace=true,
    breakautoindent=true,
    showspaces=false,
    showstringspaces=false,
    showtabs=false,
    tabsize=4,
    captionpos=b,
    belowskip=1em,
    belowcaptionskip=2em
}

\newpage

\section{Testing}

The following programs were used to test the correctness of the Y86 ISA implementation.

\subsection{Program 1}

This program initializes \%rsp to 100 and \%rax to 0xabc. 
Then \%rax is pushed onto the stack. 
Then \%rax is set with a new value 0xdef which is then pushed onto the stack. 
\%rax is set to 0xabcdeff. The top of the stack is popped into \%rcx. 
Similairly, the top of the stack is popped into \%rdx. 
The memory address 0x10 holds the value of \%rcx. 
Then that value is loaded into \%rdi. 
Then the pointer jumps to the section, newpos. 
Here, the value of \%rdi is moved to \%r12. 

\begin{lstlisting}[language=myassembly]
    .pos 0
    irmovq $100, %rsp
    irmovq $0xabc, %rax
    pushq %rax
    irmovq $0xdef, %rax
    pushq %rax
    irmovq $0xabcdeff, %rax
    popq %rcx
    popq %rdx
    rmmovq %rcx, 0x10
    mrmovq 0x10, %rdi
    jmp newpos
    nop
    nop
    rrmovq %rdi, %r11
newpos:
    rrmovq %rdi, %r12
    halt
\end{lstlisting}

This program verifies the following instructions:
\begin{itemize}
    \item irmovq
    \item pushq
    \item popq
    \item rmmovq
    \item mrmovq
    \item jmp
    \item nop
    \item rrmovq
    \item halt
\end{itemize}

The expected output is:

\begin{figure}[H]
    \centering
    \includegraphics[width=0.8\textwidth]{./images/test1_out.png}
    \caption{Expected output for Program 1}
\end{figure}

The final values of the registers and memory are as follows:

\begin{minipage}{0.3\textwidth}
    \begin{figure}[H]
        \centering
        \includegraphics[width=0.5\textwidth]{./images/test1_reg1.png}
        \caption{Final register values for Program 1}
    \end{figure}
\end{minipage}
\begin{minipage}{0.5\textwidth}
    \begin{figure}[H]
        \centering
        \includegraphics[width=0.3\textwidth]{./images/test1_reg2.png}
        \caption{Final register values for Program 1}
    \end{figure}
\end{minipage}
\begin{minipage}{0.3\textwidth}
    \begin{figure}[H]
        \centering
        \includegraphics[width=0.8\textwidth]{./images/test1_mem.png}
        \caption{Final memory values for Program 1}
    \end{figure}
\end{minipage}

\subsection{Program 2}

\%rax is set to 0. \%rbx is set to 10. 
\%rcx is set to 1. 
Then, a loop runs until \%rcx decrements to 0. 
Then \%rdi is set to 20. 

\begin{lstlisting}[language=myassembly]
    irmovq $0, %rax          # Load immediate 0 into %rax
    irmovq $10, %rbx         # Load immediate 10 into %rbx
    irmovq $1, %rcx          # Load immediate 1 into %rcx (loop counter)
    jmp loop

label:
    irmovq $20, %rdi
    jmp end

loop:
    addq %rbx, %rax          # Increment %rax by %rbx
    rrmovq %rax, %rdx        # Move %rax to %rdx

    irmovq $1, %rsi          # Load immediate 1 into %rsi
    subq %rsi, %rcx          # Decrement %rcx by 1
    irmovq $0, %rsi          # Reset %rsi to 0 for comparison

    # Use register value directly to control the loop: simple decrement and stop condition
    rrmovq %rcx, %rsi        # Copy %rcx to %rsi
    je label                 # Continue loop if %rcx is not equal to %rsi (i.e., not zero)

end:
    halt                     # Stop the program

\end{lstlisting}

This program verifies the following instructions:
\begin{itemize}
    \item irmovq
    \item addq
    \item rrmovq
    \item subq
    \item jne
    \item halt
\end{itemize}

The expected output is:

\begin{figure}[H]
    \centering
    \includegraphics[width=0.8\textwidth]{./images/test3_out.png}
    \caption{Expected output for Program 2}
\end{figure}

The final values of the registers are as follows:

\begin{figure}
    \centering
    \includegraphics[width=0.2\textwidth]{./images/test3_reg.png}
    \caption{Final register values for Program 2}
\end{figure}

\subsection{Program 3}

\%rsp (the stack pointer) is set to 0x200 . 
\%rax is set to 10. 
\%rax is stored to memory using \%rsp as the base. 
\%rdx is set to 5. The value from memory is loaded into \%rbx. 
\%rax and \%rdx is pushed onto the stack. 
The top of the stack is popped into \%rcx. 
The same is done to \%rsi. 
\%rcx increments by the value of \%rsi. 
\%rcx is moved to \%rbx. 
\%rbx and \%rcx are pushed onto the stack. 
The top of the stack is popped to \%rdi. 
The value in \%rdi is doubled then pushed to the stack. 

\begin{lstlisting}[language=myassembly]
    # Initialize stack pointer
    irmovq $0x200, %rsp    
    
    # Initialize registers
    irmovq $10, %rax       # Load immediate 10 into %rax
    rmmovq %rax, 8(%rsp)  # Store %rax at memory address 8(%rsp)
    irmovq $5, %rdx        # Load immediate 5 into %rdx
    
    # Use memory and stack operations
    mrmovq 8(%rsp), %rbx  # Load %rbx from memory address 8(%rsp)
    pushq %rax             # Push %rax onto the stack
    pushq %rdx             # Push %rdx onto the stack as well
    
    # Perform arithmetic on the stack's top values
    popq %rcx              # Pop the top of the stack into %rcx (was %rdx)
    popq %rsi              # Pop the next top of the stack into %rsi (was %rax)
    addq %rsi, %rcx        # Add %rsi to %rcx, store the result in %rcx
    rrmovq %rcx, %rbx      # Move result back into %rbx for further use
    
    # More stack manipulations
    pushq %rbx             # Push the result back onto the stack
    pushq %rcx             # Push %rcx onto the stack again
    
    # Double the value at the top of the stack
    popq %rdi              # Pop the top of the stack into %rdi
    addq %rdi, %rdi        # Double the value in %rdi
    pushq %rdi             # Push the doubled value back onto the stack
    
    # Clean up and repeat operations
    nop                    # No operation
    halt                   # Stop the program
\end{lstlisting}

This program verifies the following instructions:
\begin{itemize}
    \item irmovq
    \item rmmovq
    \item mrmovq
    \item pushq
    \item popq
    \item addq
    \item rrmovq
    \item nop
    \item halt
\end{itemize}

The expected output is:

\begin{figure}[H]
    \centering
    \includegraphics[width=0.8\textwidth]{./images/test4_out.png}
    \caption{Expected output for Program 3}
\end{figure}

The final values of the registers and memory are as follows:

\begin{minipage}{0.5\textwidth}
    \begin{figure}[H]
        \centering
        \includegraphics[width=0.4\textwidth]{./images/test4_reg.png}
        \caption{Final register values for Program 3}
    \end{figure}
\end{minipage}
\begin{minipage}{0.5\textwidth}
    \begin{figure}[H]
        \centering
        \includegraphics[width=0.5\textwidth]{./images/test4_mem.png}
        \caption{Final memory values for Program 3}
    \end{figure}
\end{minipage}

\end{document}
